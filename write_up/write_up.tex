%=================20==================40==================60==================80
%                      Chinese-English App. Methodology
%=================20==================40==================60==================80
\documentclass{article}
%==============================================================================%
%                                  Packages
%==============================================================================%
\usepackage[utf8]{inputenc}
\usepackage{graphicx}
\usepackage{amsmath}
%==============================================================================%
%                                  Commands
%==============================================================================%
\newcommand{\be}{\begin{equation}}
\newcommand{\ee}{\end{equation}}
%==============================================================================%
\title{Chinese-English App. Methodology}
\author{Chris Zhang and Shane Flynn}
\date{March 2020}
\begin{document}
%==============================================================================%
\section*{To Do List}
Development is broken into short-term and long-term directions.
We are interested in optimizing our methodology and are eager to hear from
people with linguistics and education backgrouds to comment on our approach

\subsection*{Short-Term Goals}
\begin{itemize}
  \item  We need a working prototype, be able to open an indepedent window and
  call an audio file.
  \item Scoring scheme for individual cards and user interface.
  \item Optimal number of cards for the learning decks.
  \item Graphic visualization for tones on each card.
  \item Pinyin and kanji for each card.
  \item Statistics on each card and on user usage.
  \item Main deck.
\end{itemize}

\subsection*{Long-Term Goals}
\begin{itemize}
  \item Make a database for the audio files.
  \item English recordings for mandarin speakers to practice english
  pronunication.
  \item Simplified kanji.
\end{itemize}

\section*{Vocabulary Tables}
Current vocabulary available organized by level

\section*{Database}
Given our data space will not be changing (name-english name-chinese Pinyin
tones audio) we do not need to use object-oriented methods for the database and will
simply use a relational database.
For development SQLite seems ideal as it can be used directly from standard
python library and is light-weight.
It also does not require an intermetdiate we can simply make the database
directly.

In the future we could imagine increasing the dataspace to have user information
for development we will just focus on the application itself.

\section*{Card Selection Implementation}
The choice of which card to test is crucial.

\end{document}
